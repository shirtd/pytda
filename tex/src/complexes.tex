% !TeX root = ../main.tex
\section{Sensor Networks and Simplicial Complexes} % (fold)
\label{sec:complexes}

Sensor networks generally consist of a collection of nodes in some domain with some capacity for communication between them.
This capability for communication provides local information which can be integrated to reveal global information about the underlying domain.
Such techniques often rely on knowing the locations of the nodes, which is rarely the case in practice.
On the other hand, if some minimal assumptions are made about the domain itself we can verify the quality of the network without location information.

We model sensor networks as collections of points sampled from a domain satisfying some minimal assumptions.
The communication between sensors is represented by a metric on the domain which defines some radius of communication for each node.
That is, let $\D$ be our domain imbued with some metric $\dist : \D\times\D\to\R$.
Let $P\subset\D$ be a set of points sampled from $\D$, each representing a node in our sensor network.
If we assume that each sensor is capable of communicating with nodes within some distance $\alpha > 0$ we may represent the network itself as an undirected graph $G = (P, E)$ with edges for each pair of nodes within distance $\alpha$:
\[ E = \{(p, q)\in P\mid \dist(p, q)\leq \alpha\}. \]

While this representation does provide some information about the overall connectivity of the network it does not necessarily give us any sense of how well the domain itself is covered.
That is, if our goal is to determine if every point in a subset $D\subseteq\D$ of the underlying domain is within $\alpha$ of at least one sensor we must verify that $D$ is a subset of the union of the regions spanned by each node's communication radius.
Formally, let $\ball_\e(p) := \{x\in\D\mid \dist(x, p)\leq\e\}$ denote the coverage region of a node $p\in P$ at radius $\e\in\R$ and \[P^\e := \displaystyle\bigcup_{p\in P}\ball_\e(p)\] denote the region covered by all nodes in $P$, or the $\e$-offset of $P$.
We say that a subset $D$ of $\D$ is covered by $P$ at radius $\e$ if $D\subseteq P^\e$.

We now define a generalization of a graph which can be used to represent the coverage region of a given sensor network.
A \textbf{simplicial complex} $K$ is a collection of subsets, called \textbf{simplices}, of a vertex set $V$ that is closed under taking subsets.
That is, for all $\sigma\in K$ and $\tau\subset\sigma$ it must follow that $\tau\in K$.
The \textbf{dimension} of a simplex $\sigma\in K$ is defined as $\dim(\sigma) := |\sigma|-1$ where $|\cdot|$ denotes set cardinality.
The dimension of a simplicial complex $K$ is the maximum dimension of any simplex in $K$.
We note that an undirected graph $G = (V, E)$ is a 1-dimensional simplicial complex, consisting only of 0-simplices (vertices) $V$ and 1-simplices (edges) $E$.

There is one simplicial complex in particular that encodes exactly the coverage region of a sensor network.
The \textbf{\v Cech complex} of a finite collection of points $P\subset \D$ at scale $\e > 0$ is defined as
\[ \cech^\e(P) := \left\{\sigma \subseteq P\mid \bigcap_{p\in \sigma}\ball_\e(p)\neq \emptyset \right\}. \]

\textbf{TODO} $\cech^\e(P) \cong P^\e$.

While the \v Cech complex captures exactly the coverage region in question it can only be computed with precise coordinate information.
In the case when only pairwise proximity information is known we may use the \textbf{(Vietoris-)Rips complex} which is defined for a set $P$ at scale $\e > 0$ as
\[ \rips^\e(P) := \left\{\sigma \subseteq P\mid \forall p,q\in\sigma,\ \dist(p, q)\leq\e\right\}. \]

An important result about the relationship of \v Cech and Rips complexes follows from Jung's Theorem~\cite{jung01uber} relating the diameter of a point set $P$ and the radius of the minimum enclosing ball:
\begin{equation}\label{eq:jung_inclusion}
  \cech^\e(P) \subseteq \rips^\e(P) \subseteq \cech^{\jungd \e}(P),
\end{equation}
where the constant $\jungd = \sqrt{\frac{2d}{d+1}}$ (see~\cite{buchet15efficient}).

As we will see the Rips complex may be used to verify coverage of a domain satisfying some minimal assumptions when we allow the sensors in our network to communicate at two radii $\alpha$ and $\beta$.
In short, if the the inclusion of Rips complexes at scales $\alpha < \beta$ resembles the structure of a subset of the domain then the \v Cech complex at scale $\alpha$, and therefore $P^\alpha$, does as well.

\textbf{TODO} simplicial complexes as ways to discretize space.


% section complexes (end)
