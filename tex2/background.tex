% !TeX root = main.tex
\section{Background} % (fold)
\label{sec:background}

A \textbf{length space} is a metric space $(M,\dist)$ in which the distance between any two points $x,y\in M$ is equal to the infimum of the lengths of all paths from $x$ to $y$.
Throughout we will consider a compact length space $(M,\dist)$, where there is a homeomorphism from $M$ to a compact subset of $\R^d$.

To simplify notation we will interpret $M$ directly as a subset of $\R^d$, but with the metric $\dist$ on the points identified with $M$, rather than the Euclidean metric.
We define the \textbf{complement} of a compact set $A\subseteq M$ as $\comp{A} := (\R^d\cup\{\infty\})\setminus A$, where $\R^d\cup\{\infty\}$ is the compactification of $\R^d$, which we note is homeomorphic to the $d$-sphere.


If $A\subseteq M$ is endowed with weights $w_y\geq 0$ for all $y\in A$, the \textbf{weighted distance} from a point $x$ to a weighted point $y$ is defined as the power distance $\rho_y(x) :=\sqrt{\dist(x,y)^2+w_y^2}$.
Such a set is referred to as a \textbf{weighted set}.
We use weighted distances to model coverage by disks of varying radii, where larger weights correspond to smaller radii.
These weights can be used to model high dimensional noise when the weights of each point are defined to be the distance to its projection onto a domain.

Let $A\subseteq M$ be a weighted compact set.
The \textbf{weighted $k$-nearest neighbor distance} from a point $x$ to a weighted compact set $A\subseteq M$ is defined as
\[
    \dist_k(x, A) :=\inf_{K\in {A\choose k}}\max_{y\in K} \rho_y(x).
\]
where $A\choose k$ denotes the collection of $k$-element subsets of $A$.
The \textbf{weighted $(k,\e)$-offsets} of $A$ are defined as
\[
    A^\e_k := \left\{ x\in M\mid\dist_k(x,A)\leq\e\right\}.
\]
The \textbf{unweighted $\e$-offsets} of $A$ are defined as
\[
    A^\e := \left\{ x\in M\mid \min_{a\in A}\dist(x,a)\le\e\right\},
\]
Note that for any weighted set $A$, any $\e\ge 0 $, and any $k\ge 1$ we have  $A^\e_k\subseteq A^\e$.
The \textbf{coverage region} of a point $a\in A$ at scale $\e$ is denoted $\cov(a,\e) = \left\{x\in M\mid\rho_a(x)\le \e\right\}$.

\paragraph*{\textbf{Simplicial Complexes.}}
    A \textbf{simplicial complex} $K$ is a collection of subsets, called \textbf{simplices}, of a vertex set $V$ that is closed under taking subsets.
    That is, for all $\sigma\in K$ and $\tau\subset\sigma$ it must follow that $\tau\in K$.
    The \textbf{dimension} of a simplex $\sigma\in K$ is defined as $\dim(\sigma) := |\sigma|-1$ where $|\cdot|$ denotes set cardinality.
    The dimension of a simplicial complex $K$ is the maximum dimension of any simplex in $K$.
    We define a \textbf{pair of complexes} to be a pair $(K,L)$ where $K$ is a simplicial complex and $L$ is a subcomplex of $K$.

    A \textbf{graph} $G = (V,E)$ is defined to be a simplicial complex of dimension at most 1, consisting of a vertex set $V$ and an edge set $E\subseteq{V\choose 2}$.
    Given a subset $U\subseteq V$ the \textbf{induced subgraph} in $G$ by $U$ is defined $G[U] := (U, E\cap{U\choose 2})$.

    Given a graph $G = (V, E)$ a \textbf{clique} is a collection of vertices $\sigma\subseteq V$ such that for all $u,v\in\sigma$ the edge $\{u,v\}\in E$.
    The \textbf{clique complex} of $G$ is defined to be a simplicial complex with simplices for each clique in $G$.
    \[
      \clique(G) := \{ \sigma\subseteq V\mid \forall u,v\in\sigma, \{u,v\}\in E \}
    \]
    Given a pair of graphs $(G, H)$ where $H$ is a subgraph of $G$, we will denote the pair of Clique complexes as $\clique(G,H) = (\clique(G),\clique(H))$.

    The \textbf{\v Cech complex} of a finite collection of weighted points $A\subset M$ at scale $\e$ is defined as
    \[
      \cech^\e(A) := \left\{\sigma \subseteq A\mid \bigcap_{p\in \sigma}\cov(p,\e)\neq \emptyset \right\}.
    \]
    The \textbf{(Vietoris-)Rips complex} of $A$ at scale $\e$ is defined as
    \[
      \rips^\e(A) := \left\{\sigma \subseteq A \mid \{p,q\}\in\cech^\e(A) \text{ for all $p,q\in \sigma$}\right\}.
    \]
    The Rips complex is the clique complex derived from the edges in the \v Cech complex.

    An important result about the relationship of \v Cech and Rips complexes follows from Jung's Theorem~\cite{jung01uber} relating the diameter of a point set $A$ and the radius of the minimum enclosing ball:
    \begin{equation}\label{eq:jung_inclusion}
      \cech^\e(A) \subseteq \rips^\e(A) \subseteq \cech^{\jungd \e}(A),
    \end{equation}
    where the constant $\jungd = \sqrt{\frac{2d}{d+1}}$ for unweighted sets and $\jungd = 2$ for weighted sets (see~\cite{buchet15efficient}).


\paragraph*{\textbf{The $k$-Barycentric Decomposition.}} % fold
    Given a simplicial complex $K$ we define a \textbf{flag} in $K$ to be an ordered subset of simplices $\{\sigma_1,\ldots,\sigma_t\}\subset K$ such that $\sigma_1\subset\ldots\subset\sigma_t$.
    The \textbf{barycentric decomposition} of $K$ is the simplicial complex formed by the set of flags of $K$ and is defined as $\bary(K):= \left\{U\subset K\mid U\text{ is a flag of } K\right\}$.
    The vertices of the barycentric decomposition are the simplices of $K$.
    We define the \textbf{degree} of a flag $\sigma_1\subset \cdots \subset \sigma_t$ to be $|\sigma_1|$.
    The \textbf{$k$-barycentric decomposition} of a complex $S$ is defined as
    \[
      \kbary(K) := \left\{U\subset K\mid U\text{ is a flag in $K$ with } |U|\geq k\right\}.
    \]
    The $k$-barycentric decomposition of the Clique complex of a graph $G$ will be denoted
    \[
      \clique_k(G) = \kbary(\clique(G)).
    \]
    Similarly, the $k$-barycentric decomposition of the \v Cech complex of a finite point set $A$ at a scale $\e$ will be denoted
    \[
      \cech_k^\e(A) = \kbary(\cech^\epsilon(A)).
    \]

\paragraph*{\textbf{Homology and Persistent Homology.}} % (fold)
\label{par:homology_and_persistent_homology}

    Homology is a tool from algebraic topology that gives a computable signature for a shape that is invariant under many topological equivalences, in particular homeomorphisms and homotopy equivalences.
    It gives a way to quantify the components, loops, and voids in a topological space.
    It is a favored tool for applications because its computation can be phrased as a matrix reduction problem with matrices representing a finite simplicial complex.

    Throughout, we assume singular homology over a field, so the $k$th homology group $\hom_k(C)$ of a space $C$ is vector space.
    When considering the homology groups of all dimensions, we will write $\hom_*(C)$.
    We will make extensive use of relative homology.
    That is, for a pair of spaces $(A,B)$ with $B\subseteq A$, we write $\hom_*(A,B)$ for the homology of $A$ relative to $B$.

    We can also talk about the homology of a map between two spaces.
    Given two spaces $A$ and $B$ and a map $f:A\rightarrow B$, we can consider the homology of both the spaces and the map for all homology groups due to the functoriality of homology, i.e.~we have a map $f_*=H_*(A)\rightarrow H_*(B)$, which we will denote $f_*:=H_*(A\rightarrow B)$.
    Of particular interest in this work is the homology map induced by inclusion from one space to another, in which commutativity of diagrams of spaces is preserved when passed to a diagram of homology groups.

\paragraph*{\textbf{Other Topological Notions.}} % (fold)
\label{par:other_topological_notions}
    We will employ several other standard notions from topology.
    A space is \textbf{triangulable} if it is homeomorphic to a finite simplicial complex.
    Triangulability acts as a non-degeneracy condition.

    There are vector spaces dual to the homology groups called the \textbf{cohomology groups} and they are denoted with superscripts as $\hom^*(C)$ with respect to a space $C$.
    For finite-dimensional homology groups, the so-called \textbf{Universal Coefficient Theorem} implies that the $r$-dimensional homology and cohomology groups are isomorphic.
    This will allow us to switch between the two theories when it is convenient.

    The other way we will switch between homology and cohomology is by \textbf{Alexander duality} which states, in general, that for pairs of nonempty compact spaces in $\R^d\cup \{\infty\}$, their $r$-dimensional relative homology is isomorphic to their complement spaces' $(d-r)$-dimensional relative cohomology, i.e.\ $\hom_r(X,Y)\cong \hom^{d-r}(\comp{Y},\comp{X})$,.
    % The specific version we use will be discussed in Section~\ref{sec:algorithms}.


\paragraph*{\textbf{Nerves and Persistent Nerves.}} % fold
\label{par:nerves_and_persistent_nerves}
    The \v Cech complex is a special case of a general construction known as a \textbf{nerve}.
    Let $U = \{U_i \mid i \in I \}$ be a collection of sets, where $I$ is any indexing set.
    The nerve of $U$ is the simplicial complex with vertex set $I$ such that $\sigma\subseteq I$ is a simplex if and only if $\bigcap_{i\in \sigma} U_i\neq \emptyset$.
    We say that $U$ \textbf{covers} the set $\bigcup_{i\in I}U_i$ and it is a \textbf{good cover} if the intersections are empty or contractible.
    For such covers, one can relate the nerve of the cover and union using the so-called Nerve Theorem.
    Chazal and Oudot generalized the nerve theorem to the persistence setting~\cite{chazal08towards} and Sheehy extended it to $k$-coverage~\cite{sheehy12multicover}.
    This is captured in the following lemma, where $(\cech_k^\e(A, B))=(\cech_k^\e(A),(\cech_k^\e(B))$.
    \begin{lemma}\label{lem:nerve}
      For any $B\subset A \subseteq M$, if the coverage regions $\{\cov(a,\alpha)\mid a\in A\}$ form a good cover of $A_1^\alpha$ and similarly for $A_1^\beta$, $B_1^\alpha$, and $B_1^\beta$, then the following diagram commutes for all $k$ and the vertical maps are isomorphisms.
      \begin{equation*}\label{eq:nerve_diagram}
          \xymatrix{
              \hom_*(A^\alpha_k,B^\alpha_k)~\ar[r]\ar[d]^{\cong} & \hom_*(A^\beta_k,B^\beta_k)\ar[d]^{\cong} \\
              \hom_*(\cech_k^\alpha(A, B))~\ar[r] & \hom_*(\cech_k^\beta(A, B))
          }
      \end{equation*}
    \end{lemma}

    % A combinatorial construction of this fact appears in~\cite{sheehy12multicover}, but a more direct topological argument can be found in Appendix~\ref{sec:multi_cover}.

% section background (end)
