% !TeX root = ../main.tex
\section{Introduction} % (fold)
\label{sec:introduction}

  The purpose of this report is to present research and development progress on questions regarding sensor network integrity using tools from \emph{topological data analysis} (\emph{TDA}).
  The style is intended to be pedagogical, walking the reader through the main ideas, emphasizing pictures over proofs to communicate intuition.
  The primary work thus far has been to integrate several different tools and combine them with visualization to do experimental data analysis.
  We will discuss the basic theory and the principle ideas along with examples and their visualization.

  Throughout, we will assume that a sensor network is a collection of sensors, deployed in some environment with the ability to detect nearby sensors and do some other types of measurement.
  The sensors measure some quiantity, like temperature, that is assumed to be a continuous function on the domain.
  In this light, the sensor network is a finite sample from an unknown function on an unknown space.
  The neighbor information of the sensors is correlated to distances in the domain, but the locations of the sensors (i.e. as coordinates) is not known.

  It is a challenging theoretical (and practical) setting in which to work.
  One feels intuitively that there will be very strong limits on what can be computed given the strict limits on the inputs.
  However, some very strong theoretical guarantees exist for some fundamental problems in these so-called \emph{coordinate-free sensor networks}.
  Perhaps the most immediately compelling of these results is the \emph{Topological Coverage Criterion} or \emph{TCC}.
  This gives a way to extract a guarantee of sensor coverage from the neighborhood information.
  The specific assumptions and the result are summarized in some detail in Section~\ref{sec:tcc}.
  This result, first formulated by De Silva and Ghrist~\cite{desilva07coverage} was extended and simplified by Cavanna et al.~\cite{cavanna17when}.
  It gives a polynomial-time algorithm to certify coverage.

  The TCC was a major starting point for the present work, which strives to extend these methods to the analysis of scalar fields and vector fields over sensor networks.
  That is, we'd like to extract topological information about the unknown function from the network measurements.
  In particular, we'd like to identify global inconsistencies in the data, which can manifest themselves only in the presence of nontrivial network topology.
  It is assumed that such inconsistencies are an indication of either sensor errors or possibly intentional manipulation of the network.
  Cohomology, and especially \emph{persistent cohomology} gives a way to identify and localize holes in the data where errors can hide.
  Interestingly, the theory implies that these are the \emph{only} places such errors can hide.

  The basic setting is elaborated in Section~\ref{sec:complexes}, where it is explained how the neighborhood graph is augmented to form a larger discrete structure that can be used to represent continuous functions on the the unknown domain.
  This is the basic objec that one computes and visualizes throughout this work.

  The first topological tool to consider is homology.
  In Section~\ref{sec:homology}, we explain the basic principles of homolopgy adn how they relate to the complexes of Section~\ref{sec:complexes} and the sensor networks more generally.
  The main theme is that homology gives a language for describing (and counting) holes in the network in a mathematically rigorous way.
  Moreover, the holes can be ascribed some ``size'' or other quantitative information using persistent homology.
  Section~\ref{sec:homology} also presents some examples of the visualization of persistent homology in sample networks, showing both the persistent homology (as a persistence diagram) as well as a representative of the most significant hole.

  Most previous work on homological sensor networks was phrased in terms of homology.
  However, there is a dual theory of cohomology that is in many senses, the more natural language for expressing and studying functions on the domain.
  Section~\ref{sec:cohomology} gives the basic definitions of cohomology.
  One interpretation of the cohomology theory we are using is as global structure of a discrete version of differential forms.
  Seeing as differential forms are a tool for doing caculus without coordinates, it makes sense that discrete differential forms allow us to do some calculus on discrete complexes arising from sensor networks.
  There is an extensive package by researchers at the University of Illinois implementing Discrete Exterior Calculus~\cite{bell12pydec} that we have integrated into our codebase (See also the book by Grady and Polimeni~\cite{grady10discrete}).
  We give some examples of using that code to visualize cohomology.

  Cohomology and the Discrete Exterior calculus have an intimate connection to harmonic analysis.
  As a result, one can compute the so-called \emph{harmonic cocycles}.
  These are used to give embeddings of the nework into circular coordinates.
  Examples and illustrations are given in Section~\ref{sec:cohomology}.
  It is expected that these harmonic cocycles can also be used to extend (or interpolate) incomplete data across a network.

  Last, we report on an implementation of the TCC and show the resulting figures in Section~\ref{sec:tcc}.
  This code is fully integrated into the code base.
  All code is available in a private github repository at \url{https://github.com/shirtd/pytda}.

  % Insofar as there could be discontinuities (read: impossibilities), they should relate to the nontrivial cohomology of the network.  Thus, the circular coordinates expose directly the space where errors/attacks could or should appear.

% section introduction (end)
